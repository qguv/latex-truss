\documentclass[letterpaper,11pt]{article}
\usepackage[margin=1in]{geometry} % letter paper margins

\usepackage{fancyref} % classy figure, section & table references
\usepackage{xspace}   % correcting intransitive command spacing
\usepackage{setspace} % smart commands to modify line spacing
\usepackage{leipzig}  % leipzig glossing rules
\usepackage{gb4e}     % fancy gloss environment
\usepackage{tipa}     % international phonetic alphabet

% italicize foreign text, including the first line of a gloss
% n.b. \it != \em
\newcommand{\ipa}[1]{\textit{\textipa{#1}}}
\renewcommand{\eachwordone}[1]{\textit{\textipa{#1}}}

% refs to examples are in parens; use \rex
\newcommand{\rex}[1]{(\ref{#1})}

% get rid of excessive list spacing
\let\oldexe\exe
\renewcommand{\exe}{
  \singlespacing
  \oldexe
  \setlength{\topsep}{0pt}
  \setlength{\partopsep}{0pt}
}
\let\olditemize\itemize
\renewcommand{\itemize}{
  \singlespacing
  \olditemize
  \setlength{\topsep}{0pt}
  \setlength{\partopsep}{0pt}
}
\let\oldenumerate\enumerate
\renewcommand{\enumerate}{
  \singlespacing
  \oldenumerate
  \setlength{\topsep}{0pt}
  \setlength{\partopsep}{0pt}
}

\title{\huge\bf A `Patterns' Homework Assignment}
\author{Quint Guvernator}

\begin{document}
\maketitle
\doublespacing

\section{A Topic}

Odia (formerly Oriya)\footnote{
  In November 2011, the state government voted to change the anglicized name of the language \emph{Oriya} and state \emph{Orissa} to \emph{Odia} and \emph{Odisha} respectively.
}
is an Eastern Indo-Aryan language, a distinction it shares with Bengali and Assamese. Odia is spoken by 32--33 million people in the Indian state Odisha (formerly Orissa).

\subsection{Some Glosses}

In \rex{wasting-paper} below, the child's actions are deemed bad, and the speaker expresses her disapproval by using the negative-connotation article \ipa{-\:ta}.
The article is not restricted to people, however; other inanimate referants can also take \ipa{-\:ta} as demonstrated in \rex{bad-road}.

\begin{exe}
  \ex \label{wasting-paper}
  \gll \textbf{pila-\:ta} bOhut kagOdZO nOstO kOr-utSv\super{h}-i.
  \\ child-\Art{} much paper waste do-\Prog-\Third\Sg
  \\ \trans `The child is wasting much paper.'

  \ex \label{bad-road}
  \gll \textbf{rasta-\:ta} bhOlO nuh-\~e. au \:tikie aste ga\:ri cO\:l-a-O.
  \\ road-\Art{} good be:\Neg-\Third\Sg{} more a.little slowly car run-\Caus-\Second\Pl:\Imp
  \\ \trans `The road is not good. Please drive a little more slowly.'
\end{exe}

%FIXME: replace `hw00' below with the appropriate filename
\bibliography{hw00}{}
\bibliographystyle{plain}

\end{document}
